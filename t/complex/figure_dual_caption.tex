% The data model needs to recognize dual captions in the same {figure} float.
% and produce two "logical figures" inside one physical float. Or something equivalent.
\documentclass{article}
\usepackage{graphicx}
\def\imgfile{../graphics/none.png}
\begin{document}
% dual figures in a single {figure}
\begin{figure}
  \centering
  \begin{minipage}{.5\textwidth}
    \centering
   \includegraphics[width=0.95\linewidth]{\imgfile}
   \caption{Left figure.}
   \label{fig:dual-left}
  \end{minipage}%
  \begin{minipage}{.5\textwidth}
    \centering
   \includegraphics[width=0.95\linewidth]{\imgfile}
   \caption{Right figure.}
   \label{fig:dual-right}
  \end{minipage}
  \end{figure}
% dual tables in a single {table}
\begin{table}
  \centering
  \begin{minipage}{.5\textwidth}
    \centering
   \begin{tabular}{ccc}
   a & b & c \\
   d & e & f \\
   \end{tabular}
   \caption{Left table.}
   \label{table:dual-left}
  \end{minipage}%
  \begin{minipage}{.5\textwidth}
    \centering
    \begin{tabular}{ccc}
      1 & 2 & 3 \\
      4 & 5 & 6 \\
      \end{tabular}
   \caption{Right table.}
   \label{table:dual-right}
  \end{minipage}
\end{table}

Ref test:
Figure \ref{fig:dual-left} and Figure \ref{fig:dual-right}.
Table \ref{table:dual-left} and Table \ref{table:dual-right}.

\end{document}