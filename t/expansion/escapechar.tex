\documentclass{article}
\begin{document}
\def\Z{Snore, as a macro}
\catcode`\Z=13\relax
\def Z{Snore as an active character}

% Do something like LaTeX's \@if
\makeatletter
\def\chomper#1#2#3{%
  \count@\escapechar \escapechar#1\relax
  \csname\expandafter\@gobbletwo\string#2%
    \expandafter\@gobbletwo\string#3\endcsname%}
  \escapechar\count@}
\makeatother

In \TeX, control sequences are stored \emph{without} the slash,
but  active characters have a bit set to make them distinct.

Macro: \Z.

Char: Z.

The backslash is actually the character \verb|\escapechar|, usually \verb|\| or none if negative.
That backslash is recognized in the Mouth when reading Tokens but removed from
the internal form of control sequence. It is added back by \verb|\string|. 


If LaTeXML wants to keep the backslash internally, at least it has to make these work:

\def\foobar{Foo/Bar}
\def\AfooAbar{A Foo/A Bar}
Disabled backslash gives Foo/Bar : \chomper{-1}\AAfoo\AAbar.

Normal backslash gives A Foo/A Bar: \chomper{`\\}\AAfoo\AAbar.

\end{document}
