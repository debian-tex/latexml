\documentclass{article}
\begin{document}
\section{Verbatim Environment}
Normal {\bf Bold} stuff.

Verbatim 1:
\begin{verbatim}
AbNormal {\bf NonBold} stuff.
\end{verbatim}

Verbatim 2:
\begin{verbatim}
inside verbatim
\end {verbatim}
still inside verbatim
\end{verbatim}

Verbatim 3:
\begin{verbatim} keep this
AbNormal {\bf NonBold} stuff.
keep this too    \end{verbatim} outside verbatim
More Normal {\bf Bold} stuff.

% NOT allowed
%\section{Inline \verb|\verb| verbatim.}
\section{Inline verb command}

Some inline \verb|\verb| verbatim.

More \verb|a } b # c ^ d _ e $ f| stuff;

And more \verb*|a } b # c ^ d _ e $ f| stuff

Even in math $\verb|a_b|^2$.

% This IS allowed!
\hbox{More inline \verb|\verb| verbatim.}

% and this:
x\hbox to 2in  {\bf x\verb|\verb|y.}y

% Also, make sure \verb is using the Gullet API to obtain any \expandafter tokens
% AND it is using the correct delimiter activation sequence
\def\nested{Nested | is allowed}
\expandafter\verb\expandafter|\nested .|

Note that spaces after a control-sequence are skipped on the \emph{next} read,
by which time catcodes may have been changed.
Prior to April, 2020, a space following \verb|\verb| would -- surprisingly -- have been treated as
the delimiter! Since then, spaces are skipped and the following char is used as delimiter.
We'll adopt the newer approach.

\verb |a}b#c^d_e$f| stuff

{\verb Trick-roll} T }

{\verb* Trick-roll} T }

% Another tricky case
\makeatletter
\def\verbatimlisting#1{%
  \begingroup\@verbatim
  \input#1
  \endgroup}
\makeatother
Input snippet \verbatimlisting{snippet} after.

\end{document}
