\documentclass{article}
\usepackage[acronyms]{glossaries}
\makenoidxglossaries
%\tracingmacros=2\relax
%\tracingcommands=2\relax
%\newglossaryentry{poo}{name={poo},description={doodoo}}
%\tracingmacros=0\relax
%\tracingcommands=0\relax
\newglossaryentry{potato}{%
  name={potato},
  plural={potatoes},
  description={starchy tuber}}
\newglossaryentry{cabbage}{%
  name={cabbage},
  description={vegetable with thick green or purple leaves}}
\newglossaryentry{carrot}{%
  name={carrot},
  description={orange root}}
\newglossaryentry{nothing}{%
  name={nothing},
  description={this entry is never used}}
\longnewglossaryentry{fishage}{%
  name={Fish Age}}{%
A common name for the Devonian geologic period
spanning from the end of the Silurian Period to
the beginning of the Carboniferous Period.
This age was known for its remarkable variety of
fish species.
}
\setacronymstyle{long-short}
\newacronym{svm}{SVM}{support vector machine}
\begin{document}
\section{Using Acronyms}
Acronym first use: \gls{svm}. Second use: \gls{svm}. A use \gls{svm} within a sentence.

\section{Using Glossaries}
Glossary use, various cases:
\begin{itemize}
\item Chop the \gls{cabbage}, \glspl{potato} and \glspl{carrot}.
\item Or, more loudly: Chop the \Gls{cabbage}, \Glspl{potato} and \Glspl{carrot}.
\item Or, more loudly: Chop the \GLS{cabbage}, \GLSpl{potato} and \GLSpl{carrot}.
\item For a small appetite, use only one, \gls{potato} and \gls{carrot}.
\end{itemize}
But do not use the \gls{fishage}; that's someething entirely different.
However, how about \glssymbol{fishage}?

\def\test#1{Word: #1; \xmakefirstuc{#1}; \MakeTextUppercase{#1}}%
\test{foo}.
\test{albatross}.

\printnoidxglossaries
\end{document}
